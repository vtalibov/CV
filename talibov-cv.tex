\documentclass[margin]{res}

\setlength{\textwidth}{5in}

\usepackage{hyperref}
\hypersetup{colorlinks=true, urlcolor=blue}

\usepackage[sorting=ydnt,style=numeric, giveninits=true, maxbibnames=40]{biblatex}
\addbibresource{talibov-cv.bib}

\begin{document}

\moveleft.5\hoffset\centerline{\large\bf Vladimir O. Talibov}
\moveleft\hoffset\vbox{\hrule width\resumewidth height 1pt}\smallskip
\moveleft.5\hoffset\centerline{Last update: \today}
\moveleft.5\hoffset\centerline{Svartb\"acksgatan 19 \hfill \href{https://vtalibov.xyz}{https://vtalibov.xyz}}
\moveleft.5\hoffset\centerline{SE 75332 \hfill \href{mailto:mail@vtalibov.xyz}{mail@vtalibov.xyz}}
\moveleft.5\hoffset\centerline{Uppsala, Sweden \hfill }

\begin{resume}
  
    \section{Summary}{A protein biochemist with an interest in ligand discovery. Experienced in biophysical and kinetic methods.}

\section{Skills}
{\sl \underline{Experimental:}} Biophysical methods (SPR biosensors, TSA, MST), protein techniques, expression\&purification, macromolecular crystallography\\
{\sl \underline{Computer:}} Linux, RDKit,  KNIME\\
{\sl \underline{Languages:}} English, Russian, Swedish (basic)\\
{\sl \underline{Expertise:}} Biophysical methods, enzymology, small molecules.

\section{Experience} 

{\sl Senior Scientist} \hfill Mar 2021 -- current \\
Sprint Bioscience AB, Huddinge, Sweden
\begin{itemize}
\setlength\itemsep{0em}
\item Support of FBLD projects as a member of Protein Science Team
\item Structural biology (MX)
\item Protein biochemistry.
\end{itemize}

{\sl Researcher} \hfill Aug 2019 -- Feb 2021 \\
MAX IV Laboratory, Lund, Sweden
\begin{itemize}
\setlength\itemsep{0em}
\item Development of operation protocols for MAX IV fragment screening facility
\item Design, curation and maintenance of in-house fragment library
\item BioMAX user support as a beamline scientist, including on-call service.
\end{itemize}

{\sl PhD student, Researcher} \hfill Apr 2014 -- Jun 2019 \\
Uppsala University, Uppsala, Sweden
\begin{itemize}
\setlength\itemsep{0em}
\item Development of biophysical and enzymatic assays for ligand discovery
\item Maintenance of biosensors and chromatographic equipment.
\item Teaching (20\%, MSc-level courses)
\end{itemize}

{\sl Laboratory Assistant} \hfill Jul 2012 -- Feb 2014 \\
OOO "Biochip-IMB", Moscow, Russia
\begin{itemize}
\setlength\itemsep{0em}
\item Development and validation of multiplex clinical diagnostics assays
\item QC of proteins, synthetic oligonucleotides and reactive small molecules.
\end{itemize}
 
\section{Education} {\sl PhD in Biochemistry} \hfill 2014 -- 2019 \\
Uppsala University, Uppsala, Sweden \\
Advisor: Prof. U. Helena Danielson \\
\href{http://uu.diva-portal.org/smash/record.jsf?pid=diva2%3A1256395&dswid=214}{"Interaction kinetic analysis in drug design, enzymology and protein research"}.

{\sl BSc\&MSc in Chemistry} \hfill 2008 -- 2013 \\
Moscow State University, Moscow, Russia \\
Specialisation in bioorganic chemistry.
                
\section{Interests} Molecular recognition, early stage lead discovery, screening techniques.

\section{Publications}
\nocite{*}
Research articles: 9; details are available at \href{https://scholar.google.com/citations?hl=sv&user=H5uK2zsAAAAJ&view_op=list_works&authuser=2&sortby=pubdate}{GScholar}.\\
Other: reviews - 2, book chapters - 1, patents  - 1.\\
\printbibliography[heading=none]
\end{resume}

\end{document}
